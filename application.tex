
\section{\label{sec:application} Application of the criteria to the controller parameters estimation problem}

\subsection{Smallest singular value criterion}
The application of this criterion to the controller parameters estimation problem can be justified by the fact that in the eigenparameter space the bias and variance of the estimated parameters $\hat{\rho}_{V}\tran(N)$ of the proposed problem behave as described in \autoref{subsec:ssv}.
To elucidate this fact 200 Monte Carlo experiments were performed in an open loop experiment with a step as system input.
The parameters were estimated increasing the number of samples with the VRFT method.
The bias and variance of the eigenparameters were calculated and are presented in \autoref{fig:bias_var_ol_val}.
\begin{figure}[h!]
  \centering
  \def\svgwidth{\columnwidth}
  {\footnotesize\import{images/pdf/}{bias_var_ol_val.pdf_tex}}
  \caption{\label{fig:bias_var_ol_val} Bias and variance of $\hat{\rho}_{V1}(N)$ solid line and $\hat{\rho}_{V2}(N)$ dashed line.}
\end{figure}

It is possible to see that the bias of the eigenparameter related to the smallest singular value $\hat{\rho}_{V2}(N)$ suffers the most with time, while the bias of $\hat{\rho}_{V1}(N)$ is approximately constant and close to zero.
Whereas, the variance of the two parameters decreases with the number of samples.

In order to apply the smallest singular value criterion to the parameters estimation problem and to allow a comparison between the methods, the following adjustment was made:
the smallest singular value method was applied to the regressor matrix $\Phi(t)$ defined in \eqref{eq:reg_mat_vrft} formed by the VRFT regressor vectors defined in \eqref{eq:reg_vector_vrft} as
\[
  \underline{\sigma}^2(\Phi(t)) - \underline{\sigma}^2(\Phi(t-1)) < \eta_c
\]
The informative subset is delimited while the inequality is not satisfied, this way, after the inequality is satisfied all remaining data are discarded.
% That means that the regressor vectors are not concatenated.

\subsection{Condition number criterion}
In order to apply this criterion to the parameters estimation problem the condition number method was applied to the VRFT information matrix $P(N)$ in \eqref{eq:inf_mat_vrft} as
\[
  \gamma(t) = \frac{\underline{\sigma}(P(N))}{\overline{\sigma}(P(N))} > \eta_n
\]
The informative subset is defined while the inequality is satisfied, thus, the remaining data is discarded when the inequality is no longer satisfied.

\subsection{Proposed criterion}

The proposed criterion is still in the first steps and a proper mathematical formulation needs to be developed.
This criterion was derived from graphical analysis of the reciprocal condition number $\gamma(t)$ and the estimated cost function $J_y(N)$.
The $\jy$ estimated is defined as
\begin{equation}
	\jy = \frac{1}{N_e} \sum \limits_{t=1}^{N_e} \left( y(t, \hat{\rho}(N)) - y_d(t) \right)^2
\label{eq:jy}
\end{equation}
where, $N_e$ is the number of samples to calculate the cost function, $y_d(t)$ is the desired output, and $y(t, \hat{\rho}(N))$ is the obtained output with the controller gains $\hat{\rho}(N)$ estimated until the sample $N$, where $N$ is increased one sample at a time.

The method searches for a good number of samples $t_f$ to truncate the data
\begin{equation}
	t_f = \arg \min_t \delta(t) \triangleq \gamma(t) - \hat{\gamma}(t)
\label{eq:proposed_method}
\end{equation}
where, $\hat{\gamma}(t)$ is a polynomial approximation of $\gamma(t)$.
Here, a sixth order polynomial was used to estimate $\hat{\gamma}(t)$.
This choice was made because that was the order that best adjusted $\hat{\gamma}(t)$ to $\gamma(t)$.

In other words, the goal is to find the index of the sample where the data deviates the most from the fitted curve towards bellow.
The sample where the minimum of $\delta(t)$ occurs corresponds to a point in the valley around the minimum of $\jy$.

It is important to notice that one should ignore the first $n_\gamma$ samples before looking for the minimum.
Therefore, $n_\gamma$ is a parameter that needs to be chosen by the designer.
% Here, $n_\gamma$ is a design parameter that accounts for the initial low correlation between $\gamma(t)$ and the closed loop performance.
