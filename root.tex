%%%%%%%%%%%%%%%%%%%%%%%%%%%%%%%%%%%%%%%%%%%%%%%%%%%%%%%%%%%%%%%%%%%%%%%%%%%%%%%%
%2345678901234567890123456789012345678901234567890123456789012345678901234567890
%        1         2         3         4         5         6         7         8

% !TeX root = root.tex
% !TeX spellcheck = en_US
\documentclass[letterpaper, 10 pt, conference]{ieeeconf}  % Comment this line out if you need a4paper

%\documentclass[a4paper, 10pt, conference]{ieeeconf}      % Use this line for a4 paper

\IEEEoverridecommandlockouts                              % This command is only needed if
                                                          % you want to use the \thanks command

\overrideIEEEmargins                                      % Needed to meet printer requirements.

%In case you encounter the following error:
%Error 1010 The PDF file may be corrupt (unable to open PDF file) OR
%Error 1000 An error occurred while parsing a contents stream. Unable to analyze the PDF file.
%This is a known problem with pdfLaTeX conversion filter. The file cannot be opened with acrobat reader
%Please use one of the alternatives below to circumvent this error by uncommenting one or the other
%\pdfobjcompresslevel=0
%\pdfminorversion=4

% See the \addtolength command later in the file to balance the column lengths
% on the last page of the document

% The following packages can be found on http:\\www.ctan.org
%\usepackage{graphics} % for pdf, bitmapped graphics files
%\usepackage{epsfig} % for postscript graphics files
%\usepackage{mathptmx} % assumes new font selection scheme installed
%\usepackage{times} % assumes new font selection scheme installed
%\usepackage{amsmath} % assumes amsmath package installed
%\usepackage{amssymb}  % assumes amsmath package installed

%% My packets
\usepackage{color}
\usepackage{amsmath}
\usepackage{import}
\usepackage{graphicx}
\usepackage{hyperref}
\usepackage{amssymb}
\usepackage{commath}
\usepackage{siunitx}
\usepackage{booktabs}
\usepackage{gensymb}

\hypersetup{colorlinks=true, linkcolor=black}

%% My commands
\newcommand{\tran}{^\text{T}}
\newcommand{\inv}{^{-1}}
\newcommand{\jvr}{J^\text{VR}(\rho)}
\newcommand{\ejw}{e^{j\omega}}
\newcommand{\jy}{J_y(N)}

\title{\LARGE \bf
Extraction of informative data subsets for use \\ in data-driven control
}


\author{Cristiane Silva Garcia$^{1}$ and Alexandre Sanfelice Bazanella$^{1}$% <-this % stops a space
\thanks{$^{1}$The authors are with the department of Automation and Energy -- Federal University of Rio Grande do Sul. Av. Osvaldo Aranha 99 -- CEP:90035-190 -- Porto Alegre, RS -- Brazil
        {\tt\small \{cristiane.garcia, bazanella\}@ufrgs.br}}%
\thanks{This study was financed in part by the Coordena\c{c}\~{a}o de Aperfei\c{c}oamento de Pessoal de N\'{i}vel Superior - Brasil (CAPES) - Finance Code 001.}
}

\begin{document}
\bstctlcite{IEEEexample:BSTcontrol}


\maketitle
\thispagestyle{empty}
\pagestyle{empty}


%%%%%%%%%%%%%%%%%%%%%%%%%%%%%%%%%%%%%%%%%%%%%%%%%%%%%%%%%%%%%%%%%%%%%%%%%%%%%%%%
\begin{abstract}


Performing a specific experiment to collect data to tune a controller's parameters can be a very costly and sometimes undesired task.
Instead, an attractive idea is to use data gathered from normal operation routines. In many cases these data may have enough information to estimate the optimal controller parameters, and it is important to determine for which subsets of data this is the case.
This problem has been studied in the system identification framework, where metrics have been proposed to search for informative subsets of data from a given set of routine operating data. The goal of the present work is to apply to the controller's estimation problem these metrics already proposed in the system identification literature, and also to propose a new metric. In order to attest the feasibility of the proposed solutions and compare their performance, case studies are also presented.


\end{abstract}

%%%%%%%%%%%%%%%%%%%%%%%%%%%%%%%%%%%%%%%%%%%%%%%%%%%%%%%%%%%%%%%%%%%%%%%%%%%%%%%%
\section{INTRODUCTION}


In general, within the optimal data-driven control framework, the task of estimating the controller's parameters requires the execution of a specific experiment in a plant to collect informative data.
This experiment requires a sufficient rich signal to be applied to the system, which, in most cases, differs from the signal that is applied to the system in normal operating mode.
In several cases, this is a very costly task and, sometimes, may be even impossible.
% Besides, as mentioned before, the required signal to better identify those parameters can be very different from the signal used in normal operation, as for example a pseudorandom binary sequence (PRBS), and because of that, it is inconvenient to apply it to the system.

Therefore, a better option is to use data collected from normal operation instead. In industrial processes, data gathered from normal operation routines are usually stored in a data base, being already available for free. Often, these data provide relevant information that can be used to identify the parameters of a controller. However this is not always the case, and the use of noninformative data may result in inappropriate tuning.

The problem of searching for informative subsets within the entire data set has already been treated within the system identification framework.
Informativity is determined by the regularity of the Fischer information matrix (FIM) and depends mostly on the spectrum and magnitude of the external signals applied to the plant; a detailed theoretical treatment is given in \cite{gevers2009identification}. The amount of information about the plant contained in a data set can be evaluated by the eigenvalues of the FIM.  In \cite{carrette1996discarding} the authors proposed a data removal technique that is used to discard the data that are strongly dependent of the noise.
The technique uses singular value decomposition (SVD) to extract the singular values of a regressor matrix.
Then the slope of the smallest singular value as a function of time is used as a metric to remove the data that does not have relevant information.
It was shown that using all the data may worsen the quality of the obtained parameters estimation because it would increase the bias of the estimated parameters.
% In the simulations results, the total mean square error (MSE) was used to evaluate the quality of the estimated parameters.

In \cite{peretzki2011data} an algorithm was developed to find relevant intervals of data to system identification within a historical data base.
The algorithm searches for variations on the input and output signals, and uses the condition number of the information matrix to determine if a sequence of data is informative enough.
Besides, combined to the two metrics mentioned above, the algorithm verifies how much the input and output signals are correlated, and uses it as a metric to define the previously selected data sequence as useful, and also as a quality indicator for this sequence.
An extension of the work in \cite{peretzki2011data} was presented in \cite{bittencourt2015algorithm}, where a forgetting factor was added to the estimation of the Laguerre model and a noise model was introduced.

Within the work presented in \cite{shardt2014segmentation} the algorithm proposed in \cite{peretzki2011data} was used to segment the data, aiming to detect when the process model changed, and to identify the different models for the plant corresponding to the segments found.
% However, it was shown that this approach presents over-segmentation of the data set.
% In order to minimize this effect, an approach based on the calculation of the entropy of the input and output signals was suggested.
In \cite{shardt2013data} the authors used the condition number of the Fischer information matrix to determine, from the amount of data collected from normal operating routines, the sequences of data that are relevant to identify the system model.
The same work also suggested a value for the threshold of the condition number.
In the work developed in \cite{arengas2017searching} a new method to search for informative data addressed to system identification was presented.
In that work the reciprocal condition number is also used as a metric to determine the informative subset of data.
In that case, only a few input changes are considered, and it was shown that the bias of the estimated parameters decreased using the selected subset.
In \cite{arengas2017search} an extension of \cite{arengas2017searching} to the multiple-input multiple-output (MIMO) case is presented.
In \cite{wang2018searching} a criterion to search for informative subsets in data gathered from normal operation routines is presented.
This criterion is based on finding significant magnitude changes in the input and output signals.
In the work developed in \cite{bitmead2017subspace} a rank test was presented to delimit the informative data subset.
In that case, the system is identified using subspace system identification.

In the data driven control framework this search for informative intervals of data is a issue that has not received much attention.
%Besides that, it would be good to define some requirements, as for example, the amount of data to be collected and used in the parameters estimation.Moreover, it is useful to know if the amount of data used for estimation of the parameters can interfere in the quality of the obtained estimation.
With that in mind, the present work explores the employment of the smallest singular value and the reciprocal condition number of the information matrix to the controller's parameters estimation problem.
This was inspired by the works developed in \cite{carrette1996discarding} and \cite{bittencourt2015algorithm}, which addressed the system identification problem.
Besides that, a new metric is suggested in this work, based also in the reciprocal condition number.
% However, this new metric is still in the first steps, and a proper mathematical formulation needs to be developed.

%ESTA PARTE FICOU MEIO DESLOCADA, MAS VAMOS USA-LA NA VERSAO FINAL. An advantage of using the data-driven approach, as the name suggests, is that the controller is estimated directly from the data.
%Thus, in the case of using a simple controller class as proportional-integral (PI) or a proportional-integral-derivative (PID) reduces the complexity of the information matrix compared to the system identification approach where, in general, a high order process is modeled.



%In this paper, the parameters are calculated using the Virtual Reference Feedback Tuning (VRFT) method, so the information matrix and the regressor vector used are generated by this method.
%The VRFT is a non-iterative data-driven method, which means that, in the ideal case, the data of a single experiment is enough to calculate the controller's parameters \cite{bazanella2011data}.
%This way, it is suitable to be used with normal operation data.

These metrics  are used to delimit the informative amount of data to be used to identify the controller's parameters.
Moreover, the cost function that measures closed-loop performance is used as a measure to investigate the effect of reducing the number of samples on the estimation of the parameters.
Simulation results are presented for different case studies in different scenarios in order to demonstrate the feasibility of the proposed solutions.

This paper is organized as follows: \autoref{sec:preliminaries} presents a briefly review of the methodologies proposed and presents the VRFT method.
In \autoref{sec:application} it is explained how the metrics are applied to the addressed problem.
The results of some simulation experiments are presented in \autoref{sec:experiments}.
Finally, the conclusion and the future work are discussed in \autoref{sec:conclusions}.



% !TeX root = root.tex
% !TeX spellcheck = en_US

\section{\label{sec:preliminaries} PRELIMINARIES}

\subsection{Smallest singular value approach}

The aim of this section is to provide a brief explanation of the data discarding criterion developed in \cite{carrette1996discarding}.
In that work, all the theoretical formulation was designed considering that: the system is single-input single-output (SISO) and stable, all data is from open loop simulations, and the experiment input comprises only a few step changes and long dwell times.

The system to be modeled is an auto-regressive with exogenous input (ARX), and its output $y$ and output predictor $\hat{y}(\theta)$ are given in vector form as
\begin{align}
	y &= \Phi\theta_0 + \epsilon \label{eq:system_vec}\\
	\hat{y}(\theta) &= \Phi\theta \label{eq:predic_output_vec}
\end{align}
where, $y = [y(1), \dots, y(N)]\tran$ is the system output vector,  $\theta_0 \in \mathcal{R}^{n_p}$ is the true parameters vector, and $\epsilon = [\epsilon(1), \dots, \epsilon(N)]\tran$ is the model error. %, and $\mu = [\mu(1), \dots, \mu(N)]\tran$ is the unmodeled noise vector.
Whereas, $\hat{y}(\theta)$ is the predicted output vector for any parameter vector $\theta$, and $\Phi = [\phi(1), \dots, \phi(N)]\tran \in \mathcal{R}^{N \times n_p}$ is the regressor matrix, where the regressor vector is given by
\begin{align*}
	\phi(t) = \left[ -y(t-1), \dots, -y(t-n_a), u(t-1), \dots, u(t-n_b) \right] \tran.
\end{align*}

The parameters are estimated through the standard least squares prediction error criterion given by
\[
	\hat{\theta}(N) \triangleq \arg \min_{\theta \in \mathcal{R}^{n_p}} J(\theta, N) = \frac{\norm{\varepsilon(\theta)}_2^2}{2N} = \frac{\norm{y-\hat{y}(\theta)}_2^2}{2N}
\]
where, $\varepsilon(\theta)$ is the prediction error in vector form and $\norm{\cdot}_2$ is the Euclidean norm.

The solution for this problem can be written using the pseudo-inverse $\Phi^\dagger $ of the regressor matrix $\Phi$ as
$	\hat{\theta} = \Phi^\dagger y $.
Therefore, applying the singular value decomposition technique in $\Phi$ one can get
\[
	\Phi = U \Sigma V\tran
\]
where, $\Sigma \in \mathcal{R}^{n_p\times n_p}$  is the singular value matrix formed by $\Sigma = \text{diag}(\sigma_1, \dots, \sigma_{n_p})$, and $\sigma_i^2 = \lambda_i(\Phi\tran \Phi) > 0$ for $i=1, \dots, n_p$.
The matrices $U$ and $V$ are orthogonal matrices, $U$ is the left singular matrix of $\Phi$ and $V$ is the right singular matrix of $\Phi$.

Roughly speaking, the SVD technique was used to make a similarity transformation, that is, to lead the $\hat{\theta}$ parameters vector to the eigenparameters space through $\hat{\theta}_V = V\tran \theta$.
Using the eigenparameters representation and based on the excitation assumption that $\sigma^2 \ll \sigma_i^2, \,\ i = 1, \dots, n_p $, that is, each eigensubspace energy $\sigma_i^2$ is much larger than the system noise power $\sigma^2$, it was shown that:
\begin{itemize}
	\item Each eigensubspace energy $\sigma_i^2$ is composed by a term due to the input excitation in that eigensubspace and by a term proportional to the noise energy and the number of samples.
	This way, if the increase of $\sigma_i^2$ is small it may be because it is affected only by the noise energy term, and because this term only grows with time it can deteriorate the parameters estimation.
	\item The mean of the eigenparameters $\hat{\theta}_{Vi}$, where $i$ represents the $i$-th component of $\hat{\theta}_V$, are nearly independent of each other. Although, some correlation between them exists due to the unmodeled noise.
	\item The bias of $\hat{\theta}_{Vi}$ may increase with the number of samples used if the input energy on that eigensubspace is not significant.
	\item In general, the variance of $\hat{\theta}_{Vi}$ decreases as the number of samples is increased.
	\item When there is a singular value that is significantly smaller then the others $\sigma_{i_{min}} \ll \sigma_i$, $i_{min} \neq i$, the eigenparameter $\hat{\theta}_{Vi_{min}}$ is the most affected by bias and variance.
\end{itemize}

As $\hat{\theta}_{Vi_{min}}$ is the most poorly estimated eigenparameter and it influences in the accuracy of the actual parameters, once the eigenparameters $\hat{\theta}_V$ lead to the actual parameters $\hat{\theta}$ through an orthogonal transformation, it is reasonable to only consider the time variations of $\sigma_{i_{min}}$.
This way, the discarding criterion is defined as
\begin{equation}
	\underline{\sigma}^2(N) - \underline{\sigma}^2(N-1) < \eta_c
\label{eq:smallest_sing_value}
\end{equation}
where, $\underline{\sigma}$ is the smallest singular value of the regressor matrix $\Phi$, $N$ is the number of samples and $\eta_c$ is a suitable threshold.
Therefore, the regressor is discarded if the inequality is satisfied.
The value of $\eta_c$ may be chosen as a few orders of magnitude greater than the smallest slope of the graph of $\underline{\sigma}^2(N)$.




\subsection{Condition number approach}
The second approach is presented in \cite{bittencourt2015algorithm} and a brief explanation of the method is given bellow.
In this case, the following assumptions are made: the system is SISO, the system can be well described with a linear model, and it is assumed that the input signal is driven by a sequence of steps with dwell times.


In that work, the parameters vector is estimated through the Recursive Least Squares (RLS) method given by
\begin{align*}
	\hat{\theta}_t &= \hat{\theta}_{t-1} + \bar{R}^{-1}(t)\phi(t)\varepsilon(t, \hat{\theta}_{t-1}) \\
	\bar{R}(t) &= \lambda\bar{R}(t-1) + \phi(t)\phi\tran(t) %\\
	% V_t(\hat{\theta}_t) &= \lambda V_{t-1}(\hat{\theta}_{t-1}) + \varepsilon(t, \hat{\theta}_{t-1})\varepsilon\tran(t, \hat{\theta}_{t})
\end{align*}
where, $\varepsilon(t, \hat{\theta}_{t-1})$ is the prediction error, $\phi(t)$ is the regressor vector, $\bar{R}(t)$ is the information matrix, and $\lambda$ is a weighting factor with $0<\lambda<1$.
In a simple way, the algorithm can be defined through the following steps:

\textbf{Search for any input step change:} First, look for steps in the input signal.
This way, each identified change in the input and output signals generates a data sequence.
A search for an informative subset occurs within each data sequence, which is the next step.

\textbf{Condition number test:} In this test, the reciprocal condition number of the information matrix $\bar{R}(t)$ is calculated increasing the number of samples and it is tested through a threshold as
\begin{equation}
	\gamma(t) = \frac{\underline{\sigma}(t)}{\overline{\sigma}(t)} > \eta_n
\label{eq:cond_number}
\end{equation}
where, $\underline{\sigma}$ is the smallest singular value, $\overline{\sigma}$ is the biggest singular value of $\bar{R}(t)$, and $\eta_n$ is an appropriate threshold.
Each data subset comprises one sequence truncated as soon as the above condition stops being true.

\textbf{Data quality: } The last test estimates how much the input and output signals are correlated, and uses it as a quality indicator for the data subset.
Because this test was not used in the current work, it is not presented here.
An explanation of this step can be found in \cite{bittencourt2015algorithm}.
If a data sequence is able to pass through all tests the subset is marked as useful.

\subsection{\label{sub:vrft} VRFT method}
The VRFT is a non-iterative data driven method, that is, in the ideal case the parameters can be estimated using the data collected from a single experiment.
In the ideal case, the following considerations are made: the controller is linearly parametrized, the ideal controller belongs to the controller class, and the system is not affected by noise \cite{bazanella2011data}.

Thus, the linearly parametrized controller can be described as $ C(q, \rho) = \rho\tran \bar{C}(q)$
where, $\rho$ is the parameters vector, $\bar{C}(q)$ is a vector representing the controller class, and $C(q, \rho)$ is the controller transfer function.

The method's goal is to solve the following problem
\begin{equation}
	\underset{\rho}{\text{min }} J_y(\rho) \triangleq \bar{E} \left[ y(t, \rho) - y_d(t)  \right]^2
\label{eq:cost_jy}
\end{equation}
where $J_y(\rho)$ is the reference performance criterion to be minimized.
Whereas, $y(t, \rho) = C(q, \rho) r(t)$ is the closed loop process output signal, and $y_d(t) = T_d(q)r(t)$ is the desired output, where $T_d(q)$ is the reference model.

The VRFT method transforms the problem of minimizing a cost function $J_y(\rho)$ into a prediction error identification of the controller $C(q, \rho)$, which consists in minimize $\jvr$.
The cost function $\jvr$ is defined as
\[
	\jvr = \bar{E} \left[ u(t) - \rho\tran \varphi(t) \right]^2
\]
where the regressor vector $\varphi(t)$ is given by
\begin{equation}
	\varphi(t) = \bar{C}(q)\bar{e}(t) = \bar{C}(q) \left( T_d\inv(q) -1 \right)y(t)
\label{eq:reg_vector_vrft}
\end{equation}

In this case, the regressor matrix is defined as
\begin{equation}
	\Phi(t) = \left[ \varphi(1), \ldots, \varphi(N) \right]\tran
\label{eq:reg_mat_vrft}
\end{equation}

Therefore, the controller's parameters $\hat{\rho}$ can be estimated by minimizing the squares of the difference between $\hat{\rho}\tran \varphi(t)$ and $u(t)$ solving the following normal equation
\begin{equation}
	\hat{\rho}(N) = \left[ \sum \limits_{k=1}^N \varphi(t) \varphi\tran(t) \right]\inv \left[\sum \limits_{k=1}^N \varphi(t) u(t) \right] \label{eq:rho_idealcase}
\end{equation}
where, $N$ is the number of samples collected in the experiment, and $\hat{\rho}(N)$ is the parameters vector estimated up to the $N$-th sample.
Now, it is possible to define the information matrix $P_N$ as
\begin{equation}
	P_N = \left[ \sum \limits_{k=1}^N \varphi(t) \varphi\tran(t) \right]
\label{eq:inf_mat_vrft}
\end{equation}

When the assumption that the ideal controller belongs to the controller class is no longer guaranteed, also known as mismatched case, the regressor vector $\varphi(t)$ and the input signal $u(t)$ are filtered by a filter defined as $L(\ejw) = T_d(\ejw) \left( 1- T_d(\ejw) \right)$.
This filter is applied to make the minimum of the cost functions close to each other.
A more detailed explanation about the approaches of the VRFT to deal with the mismatched case, the noise case, and other approaches can be found in \cite{bazanella2011data}.

In the sequel of this work an estimation for the cost function \eqref{eq:cost_jy} will be used.
This estimation $\jy$ is defined as
\begin{equation}
	\jy = \frac{1}{N_e} \sum \limits_{t=1}^{N_e} \left( y(t, \hat{\rho}(N)) - y_d(t) \right)^2
\label{eq:jy}
\end{equation}
where, $N_e$ is the number of samples to calculate the cost function, $y_d(t)$ is the desired output, and $y(t, \hat{\rho}(N))$ is the obtained output with the controller gains $\hat{\rho}(N)$ estimated until the sample $N$, where $N$ is increased one sample at a time.





\section{\label{sec:application} Application of the methods to the controller parameters estimation problem}

\subsection{Smallest singular value method}
The application of this method to the controller parameters estimation problem can be justified by the fact that in the eigenparameter space the bias and variance of the estimated parameters $\hat{\rho}_{V}\tran(N)$ of the proposed problem behave as described in \cite{carrette1996discarding}.
To elucidate this fact 200 Monte Carlo experiments were performed in an open loop experiment with a step as system input.
The parameters were estimated increasing the number of samples with the VRFT method.
The bias and variance of the eigenparameters were calculated and are presented in \autoref{fig:bias_var_ol_val}.
\begin{figure}[h!]
  \centering
  \def\svgwidth{\columnwidth}
  {\footnotesize\import{images/pdf/}{bias_var_ol_val.pdf_tex}}
  \caption{\label{fig:bias_var_ol_val} Bias and variance of $\hat{\rho}_{V1}(N)$ solid line and $\hat{\rho}_{V2}(N)$ dashed line.}
\end{figure}

It is possible to see that the bias of the eigenparameter related to the smallest singular value $\hat{\rho}_{V2}(N)$ suffers the most with time, while the bias of $\hat{\rho}_{V1}(N)$ is approximately constant and close to zero.
Whereas, the variance of the two parameters decreases with the number of samples.


In order to apply this technique to the parameters estimation problem and to allow a comparison between the methods, the following adjustment was made:
the smallest singular value method was applied to the regressor matrix $\Phi(t)$ defined in \eqref{eq:reg_mat_vrft} formed by the VRFT regressor vectors defined in \eqref{eq:reg_vector_vrft} as
\[
  \underline{\sigma}^2(\Phi(t)) - \underline{\sigma}^2(\Phi(t-1)) < \eta_c
\]
The informative subset is delimited while the inequality is not satisfied, this way, after the inequality be satisfied all remaining data is discarded.
That means that the regressor vectors are not concatenated.


% \begin{enumerate}
%   \item The method is used to determine the informative data subset
% 	\item The method is used to determinate the informative subset of data to be used in the estimation of the parameters.
%   To accomplish that, the smallest singular value method was applied to the regressor matrix $\Phi(t)$ defined in \eqref{eq:reg_mat_vrft} formed by the VRFT regressor vectors defined in \eqref{eq:reg_vector_vrft} as
%   \[
%     \underline{\sigma}^2(\Phi(t)) - \underline{\sigma}^2(\Phi(t-1)) < \eta_c
%   \]
%   The informative subset is delimited while the inequality is not satisfied, this way, after the inequality be satisfied all remaining data is discarded.
	% \item The threshold $\eta_c$ is obtained as $\eta_c = k \,\ \eta_{min}$
	% where, $k$ is a constant term determined to give the better results and $\eta_{min}$ is the smallest slope of the graph of $\underline{\sigma}^2(N)$.
% \end{enumerate}



\subsection{Application of the condition number method to the addressed problem}
In order to apply this method to the parameters estimation problem the condition number method was applied to the VRFT information matrix $P_N$ in \eqref{eq:inf_mat_vrft} as
\[
  \gamma(t) = \frac{\underline{\sigma}(P_N)}{\overline{\sigma}(P_N)} > \eta_n
\]
The informative subset is defined while the inequality is satisfied, thus, the remaining data is discarded when the inequality is no longer satisfied.

\subsection{Another proposed approach}

The proposed approach was derived from graphical analysis of the reciprocal condition number $\gamma(t)$ and the cost function $J_y(N)$ defined in \eqref{eq:jy}.
% The information matrix used is from the VRFT method and the controller's parameters are estimated also using this method.
In a simple way, the method searches for a good number of samples $t_f$ to truncate the data
\begin{equation}
	t_f = \arg \min_t \delta(t) \triangleq \gamma(t) - \hat{\gamma}(t)
\label{eq:proposed_method}
\end{equation}
where, $\hat{\gamma}(t)$ is a polynomial approximation of $\gamma(t)$.
Here, a sixth order polynomial was used to estimate $\hat{\gamma}(t)$.
This choice was made because that was the order that better adjusted $\hat{\gamma}(t)$ to $\gamma(t)$.

In other words, the goal is to find the index of the sample where the data deviates the most from the fitted curve towards bellow.
In general, the sample where the minimum of $\delta(t)$ occurs corresponds to a point in the valley around the minimum of $\jy$.

It is important to notice that one should ignore the first $n_\gamma$ samples before looking for the minimum.
Here, $n_\gamma$ is a design parameter that accounts for the initial low correlation between $\gamma(t)$ and the closed loop performance.


% !TeX root = root.tex
% !TeX spellcheck = en_US
\section{\label{sec:experiments} SIMULATION EXAMPLES}

In this section the simulation experiments and the results obtained are presented.
To each simulation example the algorithms were applied as presented in \autoref{sec:application}.
In all cases the output is affected by a colored noise $v(t)$ generated by a white Gaussian noise filtered by the following transfer function
\begin{equation*}
	H(q) = \frac{q^2 + 0.8q + 0.3}{q^2 - 0.8q}
\end{equation*}

The results obtained with each method are compared by the value obtained with the cost function $J_y(N)$ in \eqref{eq:jy}.
Therefore, to compare numerically the obtained values it was calculated how many values for the $\jy$ found remain in the smallest value of the cost function plus the difference between the smallest and the final value of the cost function as
\begin{equation}
	q = \text{min}(\jy) + \left( \frac{f(\jy) - \text{min}(\jy)}{4} \right)
\label{eq:threshold_jy}
\end{equation}
where $f(\jy)$ corresponds to the final value of the cost function.

\subsection{Open loop case}

In this simulation experiment 500 Monte Carlo simulations of an open loop experiment were performed.
A step with 200 samples was used as input $u(t)$ for the system.
The system transfer function $G(q)$ is given by
\begin{equation}
	G(q) = \frac{0.42794 (q-0.9145)}{(q-0.867) (q-0.9587)}
\label{eq:tf_system}
\end{equation}

The ideal controller $C_i(q)$ is given by
\begin{equation}
	C_i(q) = \frac{0.156q - 0.1454}{q-1}
\label{eq:tf_ci}
\end{equation}


As mentioned before the parameters were estimated through the VRFT method increasing number of samples for each realization.
The values used for the thresholds are: $\eta_c = 60\eta_{min} $ for the smallest singular value method, $\eta_n = 0.02$ for the reciprocal condition number method, and $\eta_\gamma = 15$ as initial start for the proposed method.


The confidence interval of the methods is presented in \autoref{fig:conf_exp1}.
As can be seen, independently of the method used the parameters estimated are polarized, this occurs because the simplest approach of the VRFT method is applied and the signals are affected by noise.
\begin{figure}[h!]
  \centering
  \def\svgwidth{\columnwidth}
  {\footnotesize\import{images/pdf/}{polarization_simple_ol.pdf_tex}}
  \caption{\label{fig:conf_exp1} Confidence interval of the methods.}
\end{figure}


The numeric comparative was calculated as in \eqref{eq:threshold_jy} and the values found are presented in \autoref{tab:comp_exp1}.
\begin{table}[h!]
  \caption{Comparative -- Experiment 1 \label{tab:comp_exp1}}
  \centering
  \begin{tabular}{lc}
  \toprule
  Method & Below $q$\\
  \midrule
  Proposed method & 326 (65.2\%) \\
  Reciprocal condition number & 294 (58.8\%) \\
  Smallest singular value & 118 (23.6\%) \\
  \bottomrule
  \end{tabular}
\end{table}

The closed loop response obtained for one realization randomly chosen is presented in \autoref{fig:step_simple_ol}.
It is possible to see that the obtained responses are close to the desired one, different from the obtained response when all data is used to estimate the parameters.
\begin{figure}[h!]
  \centering
  \def\svgwidth{\columnwidth}
  {\footnotesize\import{images/pdf/}{step_simple_ol.pdf_tex}}
  \caption{\label{fig:step_simple_ol} Step responses.}
\end{figure}


\subsection{Closed loop case}
In this case the same system transfer functions for $G(q)$ in \eqref{eq:tf_system} and $C_i(q)$ in \eqref{eq:tf_ci} were used.
The controller that was in the system when the experiments were performed is given by
\begin{equation*}
	C(q) = \frac{0.1469q - 0.1388}{q-1}
\end{equation*}

In this case, a step sequence was used as input reference $r(t)$, and 500 Monte Carlo simulations were performed.
Here, also for each realization, the parameters were estimated using the VRFT method with the number of samples increasing one sample each time.
The values used for the thresholds are: $\eta_c = 80\eta_{min} $ for the smallest singular value method, $\eta_n = 0.02$ for the reciprocal condition number method, and $\eta_\gamma = 35$ as initial start for the proposed method.
The confidence interval of the methods is presented in \autoref{fig:polarization_cl_deg_r}.
As can be seen the estimated parameters are polarized.
\begin{figure}[h!]
  \centering
  \def\svgwidth{\columnwidth}
  {\footnotesize\import{images/pdf/}{polarization_cl_deg_r.pdf_tex}}
  \caption{\label{fig:polarization_cl_deg_r} Confidence interval of the methods.}
\end{figure}

The numeric comparative was calculated and is presented in Table~\ref{tab:comp_cl_deg_r}.
\begin{table}[h!]
\caption{Comparative -- Experiment 2 \label{tab:comp_cl_deg_r}}
\centering
\begin{tabular}{lc}
\toprule
Method & Below $q$\\
\midrule
Proposed method & 358 (71.6\%) \\
Reciprocal condition number & 139 (27.8\%) \\
Smallest singular value & 104 (20.8\%) \\
\bottomrule
\end{tabular}
\end{table}

The cost function was calculated for four randomly selected realizations and is presented in \autoref{fig:jy_cl_deg_r}.
Each cost function was calculated with increasing number of samples and the dots indicate the ending of the intervals returned by each method.
It is possible to see that reducing the number of samples can lead to a small value for the cost function compared to the value found using all data.
\begin{figure}[h!]
  \centering
  \def\svgwidth{\columnwidth}
  {\footnotesize\import{images/pdf/}{jy_cl_deg_r.pdf_tex}}
  \caption{\label{fig:jy_cl_deg_r} Cost functions and points returned by each method of four randomly selected realizations. Each point corresponds to: $\blacktriangledown$ the proposed method, $\bullet$ the reciprocal condition number method, and $\blacklozenge$ the smallest singular value method.}
\end{figure}

% \subsubsection*{Ramp and soak}
%
%
% \subsubsection*{Perturbation input}

\subsection{Mismatched case}
This simulation example comprises the case when the ideal controller does not belongs to the controller class.
In this case, also 500 Monte Carlo simulations were performed and a step was used as system input $u(t)$.
The parameters were calculated using the VRFT approach for the mismatched case as described in \autoref{sub:vrft}.
The system transfer function $G(q)$ is given by
\[
	G(q) = \frac{0.19963 (q-0.9783)}{(q-0.8513) (q-0.965)}
\]

The controller class used to identify the controller's parameters is a PI, while the ideal controller is a PID given by
\[
	C_i(q) = \frac{0.1557 q^2 - 0.1485 q + 0.0354}{q^2 - q}
\]

The values used for the thresholds are: $\eta_c = \num{1e5}\eta_{min} $ for the smallest singular value method, $\eta_n = 0.002$ for the reciprocal condition number method, and $\eta_\gamma = 15$ as initial start for the proposed method.
The numeric comparative was calculated and is presented in Table~\ref{tab:comp_ol_mismatched}.
It is possible to see that for this case the obtained results are not so attractive.
That can also be seen through the Bode diagrams in figures \ref{fig:bode_prop}, \ref{fig:bode_cond} and \ref{fig:bode_sv}.
\begin{table}[h!]
\caption{Comparative -- Experiment 3 \label{tab:comp_ol_mismatched}}
\centering
\begin{tabular}{lc}
\toprule
Method & Below $q$\\
\midrule
Proposed method & 104 (20.8\%) \\
Reciprocal condition number & 85 (17\%) \\
Smallest singular value & 84 (16.8\%) \\
\bottomrule
\end{tabular}
\end{table}

\begin{figure}[h!]
  \centering
  \def\svgwidth{\columnwidth}
  {\footnotesize\import{images/pdf/}{bode_prop.pdf_tex}}
  \caption{\label{fig:bode_prop} Bode diagram of the controllers found using the proposed method.}
\end{figure}

\begin{figure}[h!]
  \centering
  \def\svgwidth{\columnwidth}
  {\footnotesize\import{images/pdf/}{bode_cond.pdf_tex}}
  \caption{\label{fig:bode_cond} Bode diagram of the controllers found using the condition number method.}
\end{figure}

\begin{figure}[h!]
  \centering
  \def\svgwidth{\columnwidth}
  {\footnotesize\import{images/pdf/}{bode_sv.pdf_tex}}
  \caption{\label{fig:bode_sv} Bode diagram of the controllers found using the smallest singular value method.}
\end{figure}




% \subsection{Instrumental variable}


\section{\label{sec:conclusions} CONCLUSIONS}

In this work, some methods existent in the literature aiming at finding informative subsets from data gathered from normal operation routines, originally adapted to the system identification framework, were applied to the controller's estimation problem.
The methods applied are based on the smallest singular value criterion and the reciprocal condition number criterion.
Moreover, a new criterion based on the reciprocal condition number was presented and applied to the same problem.
Simulation examples were also presented whose results indicate that using these criteria may improve the parameters estimation with respect to the use of the whole batch of routine data.

%There are still open issues as the development of a stronger mathematical proof of the applicability of the literature criteria to the controller's estimation problem.
%Also the development of a better mathematical formulation for the proposed criterion, and to extend these criteria to the multivariable case.


\addtolength{\textheight}{-12cm}   % This command serves to balance the column lengths
                                  % on the last page of the document manually. It shortens
                                  % the textheight of the last page by a suitable amount.
                                  % This command does not take effect until the next page
                                  % so it should come on the page before the last. Make
                                  % sure that you do not shorten the textheight too much.

%%%%%%%%%%%%%%%%%%%%%%%%%%%%%%%%%%%%%%%%%%%%%%%%%%%%%%%%%%%%%%%%%%%%%%%%%%%%%%%%



%%%%%%%%%%%%%%%%%%%%%%%%%%%%%%%%%%%%%%%%%%%%%%%%%%%%%%%%%%%%%%%%%%%%%%%%%%%%%%%%


%\section*{ACKNOWLEDGMENT}
%
%The authors would like to thanks the National Council for Scientific and Technological Development – CNPq/BR for the financial support.


%%%%%%%%%%%%%%%%%%%%%%%%%%%%%%%%%%%%%%%%%%%%%%%%%%%%%%%%%%%%%%%%%%%%%%%%%%%%%%%%

\bibliographystyle{IEEEtran}
\bibliography{ref}




\end{document}
