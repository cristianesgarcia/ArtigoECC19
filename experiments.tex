% !TeX root = root.tex
% !TeX spellcheck = en_US
\section{\label{sec:experiments} SIMULATION EXAMPLES}

In this section the simulation experiments and the results obtained are presented.
To each simulation example the criteria were applied as presented in \autoref{sec:application}.
In all cases the output is affected by a colored noise $v(t)$ generated by a white Gaussian noise filtered by the following transfer function
\begin{equation*}
	H(q) = \frac{q^2 + 0.8q + 0.3}{q^2 - 0.8q}
\end{equation*}

The results obtained with each criterion are compared by the value obtained with the cost function $J_y(N)$ in \eqref{eq:jy}.
To compare numerically the obtained values it was calculated how many values for the $\jy$ found remain in the smallest value of the cost function plus the difference between the smallest and the final value of the cost function divided by four, as
\begin{equation}
	q = \text{min}(\jy) + \left( \frac{f(\jy) - \text{min}(\jy)}{4} \right)
\label{eq:threshold_jy}
\end{equation}
where $f(\jy)$ corresponds to the final value of the cost function.

\subsection{Open loop case}

In this simulation experiment 500 Monte Carlo simulations of an open loop experiment were performed.
A step with 200 samples was used as input $u(t)$ for the system.
The system transfer function $G(q)$ is given by
\begin{equation}
	G(q) = \frac{0.42794 (q-0.9145)}{(q-0.867) (q-0.9587)}
\label{eq:tf_system}
\end{equation}

The ideal controller $C_i(q)$ is given by
\begin{equation}
	C_i(q) = \frac{0.156q - 0.1454}{q-1}
\label{eq:tf_ci}
\end{equation}


As mentioned before, the parameters were estimated through the VRFT method increasing number of samples for each realization.
The values used for the thresholds are: $\eta_c = 60\eta_{min} $ for the smallest singular value criterion, $\eta_n = 0.02$ for the reciprocal condition number criterion, and $\eta_\gamma = 15$ as initial start for the proposed criterion.


The confidence interval of the criteria is presented in \autoref{fig:conf_exp1}.
As can be seen, regardless of the criterion used the parameters estimated are biased.
This occurs because the simplest approach of the VRFT method is applied and the signals are affected by noise.
Moreover, it is possible to observe that the bias of the parameters obtained applying the condition number criterion and the proposed criterion is almost equal, while the variance obtained with the proposed criterion is bigger then the variance obtained with the condition number criterion.
In contrast, the smallest singular value criterion shows the highest values for bias and variance.
\begin{figure}[h!]
  \centering
  \def\svgwidth{\columnwidth}
  {\footnotesize\import{images/pdf/}{polarization_simple_ol.pdf_tex}}
  \caption{\label{fig:conf_exp1} Confidence interval of the criteria for the open loop experiment.}
\end{figure}


The numeric comparative was calculated as in \eqref{eq:threshold_jy} and the values found are presented in \autoref{tab:comp_exp1}.
It is possible to observe that the proposed criterion and the reciprocal condition number criterion show better results than the smallest singular value criterion.
\begin{table}[h!]
  \caption{Comparative -- Experiment 1 \label{tab:comp_exp1}}
  \centering
  \begin{tabular}{lc}
  \toprule
  Criterion & Below $q$\\
  \midrule
  Proposed & 326 (65.2\%) \\
  Reciprocal condition number & 294 (58.8\%) \\
  Smallest singular value & 118 (23.6\%) \\
  \bottomrule
  \end{tabular}
\end{table}


The total mean square error (MSE) was also calculated, for each criterion, as
\begin{equation}
	MSE = \sum\limits_{i=1}^{n_p}\left( \text{bias}^2(\hat{\rho}_i) + \text{var}(\hat{\rho}_i) \right)
\label{eq:mse}
\end{equation}
where $\hat{\rho}$ is the estimated controller gains and $n_p$ is the number of parameters.
The obtained values are: 0.00289 for the proposed criterion, 0.00355 for the condition number criterion, and 0.00499 for the smallest singular value criterion.


The closed loop responses, obtained with one realization randomly chosen for each method, are presented in \autoref{fig:step_simple_ol}.
It is possible to see that the obtained responses are close to the desired one, different from the obtained response when the entire data set is used to estimate the parameters.
\begin{figure}[h!]
  \centering
  \def\svgwidth{\columnwidth}
  {\footnotesize\import{images/pdf/}{step_simple_ol.pdf_tex}}
  \caption{\label{fig:step_simple_ol} Closed loop step responses for each criterion.}
\end{figure}


\subsection{Closed loop case}
In this case the same system transfer functions for $G(q)$ in \eqref{eq:tf_system} and $C_i(q)$ in \eqref{eq:tf_ci} were used.
The controller that was in the system when the experiments were performed is given by
\begin{equation*}
	C(q) = \frac{0.1469q - 0.1388}{q-1}
\end{equation*}

In this case, a step sequence was used as input reference $r(t)$, and 500 Monte Carlo simulations were performed.
Here, also for each realization, the parameters were estimated using the VRFT method with the number of samples increasing one sample each time.
The values used for the thresholds are: $\eta_c = 80\eta_{min} $ for the smallest singular value criterion, $\eta_n = 0.02$ for the reciprocal condition number criterion, and $\eta_\gamma = 35$ as initial start for the proposed criterion.
The confidence interval of the methods is presented in \autoref{fig:polarization_cl_deg_r}.
As can be seen, the bias obtained with the proposed criterion and the condition number criterion is almost equal.
In this case, applying the smallest singular value we also obtained the highest values of the bias and variance for the estimated parameters.
\begin{figure}[h!]
  \centering
  \def\svgwidth{\columnwidth}
  {\footnotesize\import{images/pdf/}{polarization_cl_deg_r.pdf_tex}}
  \caption{\label{fig:polarization_cl_deg_r} Confidence interval of the criteria for the closed loop experiment.}
\end{figure}

The numeric comparative was calculated and is presented in Table~\ref{tab:comp_cl_deg_r}.
In this case, the proposed criterion shows the best result.
\begin{table}[h!]
\caption{Comparative -- Experiment 2 \label{tab:comp_cl_deg_r}}
\centering
\begin{tabular}{lc}
\toprule
Method & Below $q$\\
\midrule
Proposed method & 358 (71.6\%) \\
Reciprocal condition number & 139 (27.8\%) \\
Smallest singular value & 104 (20.8\%) \\
\bottomrule
\end{tabular}
\end{table}

The MSE was also calculated, for each criterion, as in \eqref{eq:mse} and the obtained values are: 0.00640 for the proposed criterion, 0.00681 for the condition number criterion, and 0.02295 for the smallest singular value criterion.

The cost function was calculated for four randomly selected realizations and is presented in \autoref{fig:jy_cl_deg_r}.
Each cost function was calculated with increasing number of samples and the dots indicate the ending of the intervals returned by each criterion.
It is clear that using only a subset of the data results in a smaller value for the cost function compared to the value found using the whole data set.
\begin{figure}[h!]
  \centering
  \def\svgwidth{\columnwidth}
  {\footnotesize\import{images/pdf/}{jy_cl_deg_r.pdf_tex}}
  \caption{\label{fig:jy_cl_deg_r} Cost functions and points returned by each method of four randomly selected realizations. Each point corresponds to: $\blacktriangledown$ the proposed criterion, $\bullet$ the reciprocal condition number criterion, and $\blacklozenge$ the smallest singular value criterion.}
\end{figure}

% \subsubsection*{Ramp and soak}
%
%
% \subsubsection*{Perturbation input}

\subsection{Mismatched case}
This simulation example comprises the case when the ideal controller does not belong to the controller class.
In this case, also 500 Monte Carlo simulations were performed and a step was used as system input $u(t)$.
The parameters were calculated using the VRFT approach for the mismatched case as described in \autoref{sub:vrft}.
The system transfer function $G(q)$ is given by
\[
	G(q) = \frac{0.19963 (q-0.9783)}{(q-0.8513) (q-0.965)}
\]

The controller class used to identify the controller's parameters is a PI, while the ideal controller is a PID given by
\[
	C_i(q) = \frac{0.1557 q^2 - 0.1485 q + 0.0354}{q^2 - q}
\]

The values used for the thresholds are: $\eta_c = \num{1e5}\eta_{min} $ for the smallest singular value criterion, $\eta_n = 0.002$ for the reciprocal condition number criterion, and $\eta_\gamma = 15$ as initial start for the proposed criterion.
The numeric comparative was calculated and is presented in Table~\ref{tab:comp_ol_mismatched}.
It is possible to see that for this case the obtained results are not so good.
That can also be seen through the Bode diagrams in figures \ref{fig:bode_prop}, \ref{fig:bode_cond} and \ref{fig:bode_sv}.
\begin{table}[h!]
\caption{Comparative -- Experiment 3 \label{tab:comp_ol_mismatched}}
\centering
\begin{tabular}{lc}
\toprule
Method & Below $q$\\
\midrule
Proposed method & 104 (20.8\%) \\
Reciprocal condition number & 85 (17\%) \\
Smallest singular value & 84 (16.8\%) \\
\bottomrule
\end{tabular}
\end{table}

In this case, the MSE was also calculated, for each criterion, as in \eqref{eq:mse} and the obtained values are: 0.00612 for the proposed criterion, 0.00646 for the condition number criterion, and 0.04655 for the smallest singular value criterion.


\begin{figure}[h!]
  \centering
  \def\svgwidth{\columnwidth}
  {\footnotesize\import{images/pdf/}{bode_prop.pdf_tex}}
  \caption{\label{fig:bode_prop} Bode diagram of the controllers found in the Monte Carlo simulations using the proposed criterion.}
\end{figure}

\begin{figure}[h!]
  \centering
  \def\svgwidth{\columnwidth}
  {\footnotesize\import{images/pdf/}{bode_cond.pdf_tex}}
  \caption{\label{fig:bode_cond} Bode diagram of the controllers found in the Monte Carlo simulations using the condition number criterion.}
\end{figure}

\begin{figure}[h!]
  \centering
  \def\svgwidth{\columnwidth}
  {\footnotesize\import{images/pdf/}{bode_sv.pdf_tex}}
  \caption{\label{fig:bode_sv} Bode diagram of the controllers found in the Monte Carlo simulations using the smallest singular value criterion.}
\end{figure}




% \subsection{Instrumental variable}
